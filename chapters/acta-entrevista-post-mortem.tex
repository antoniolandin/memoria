\chapter{Acta de entrevista post mortem}

\textbf{ACTA DE LA ENTREVISTA POST MORTEM REALIZADA POR LOS PARTICIPANTES DE LA EMPRESA K-VATA, PARA REVISAR EL DESARROLLO DEL PROYECTO.}\\

En U-tad, a dieciséis de diciembre de dos mil veinticuatro, siendo las 19:00 horas, se reúnen los siguientes integrantes de la empresa K-VATA para debatir sobre los aspectos realizados durante el desarrollo del proyecto.

\begin{itemize}
    \item Alejandro Jiménez García
    \item Antonio Cabrera Landín
    \item Antonio Pérez Marquéz
\end{itemize}

En la reunión se realizaron las siguientes aportaciones:

\begin{enumerate}
    \item ¿Se ha trabajado correctamente durante todo el proyecto?
    \begin{enumerate}
        \item Antonio Cabrera: Si, en general se ha aplicado una buena metodología de trabajo durante todo el proyecto.
        \item Antonio Pérez: Si, desde un primer momento hemos trabajado en conjunto, discutiendo en grupo cómo hacer las partes del proyecto y que cosas no teníamos claras.
        \item Alejandro Jiménez: Si, desde un principio se decidió que todo se discutía con los demás integrantes para poder aclarar y resolver todos los problemas.
 
    \end{enumerate}
    \item ¿La organización ha estado impuesta desde un principio?
        \begin{enumerate}
            \item Antonio Cabrera: No, al principio estábamos menos organizados, pero a lo largo de las semanas hemos ido mejorando nuestra organización.
            \item Antonio Pérez: No, al comienzo no seguíamos ningún método de organización, no ha sido hasta la última entrega cuando hemos aplicado un buen método de como organizarnos el trabajo.
            \item Alejandro Jiménez: No, nos costó al principio organizarnos fue bastante complicado porque no teníamos claras todas las necesidades del cliente. Pero tras saber todo, nos organizamos y realizamos un buen trabajo.
        \end{enumerate}

    \item ¿Todos los integrantes del proyecto han trabajado adecuadamente?
    \begin{enumerate}
        \item Antonio Cabrera: Si, todos han trabajado adecuadamente.
        \item Antonio Pérez: Si, nos hemos dividido los trabajos de forma equitativa y los tres hemos trabajado una misma cantidad de tiempo, cumpliendo con las fechas y lo que teníamos que entregar.
        \item Alejandro Jiménez: Si, en los trabajos todos los integrantes hemos participado aunque hayan estado “divididos”.
    \end{enumerate}

    \item ¿Se ha entendido desde un principio lo que el cliente nos ha pedido?
    \begin{enumerate}
        \item Antonio Cabrera: No del todo, no fue hasta que tuvimos las reuniones con el cliente que entendimos realmente lo que quería.
        \item Antonio Pérez: Si, desde el principio teníamos la idea correcta de lo que quería el cliente, aunque hubieran ciertas cosas que no estaban explicadas claramente, pero estas dudas las hemos resuelto en las reuniones con el cliente.
        \item Alejandro Jiménez: A medias, desde un principio tuvimos una idea de lo que necesitaba el cliente, pero no fue hasta la primera reunión que aclaramos todas esas dudas y pudimos trabajar al cien por cien.
    \end{enumerate}

    \item ¿Ha existido una buena comunicación entre los integrantes?
    \begin{enumerate}
        \item Antonio Cabrera: Si, la comunicación entre los miembros del equipo ha sido fluida.
        \item Antonio Pérez: Si, hemos comunicado todo lo que había que hacer en cada momento, y se ha transmitido claramente quien se encarga de que parte del trabajo.
        \item Alejandro Jiménez: Si, todos los avances que se han producido durante el desarrollo del trabajo se han ido comunicando, por lo que ha habido una buena comunicación
    \end{enumerate}

    \item ¿Puedes mencionar tres cosas que salieron bien durante este proyecto?
    \begin{enumerate}
        \item Antonio Cabrera: Las entrevistas con el cliente, el ambiente en el equipo y la recogida de requisitos
        \item Antonio Pérez: La comunicación entre los miembros del grupo, el reparto de actividades entre cada miembro del equipo y la preparación a las reuniones con el cliente.
        \item Alejandro Jiménez: La buena relación que hemos tenido, que ha ayudado a que haya una buena comunicación; el desarrollo de los UML, que nos permitieron darle una mayor perspectiva al trabajo; y las presentaciones al cliente, que nos permitieron pulir nuestros errores o cambiarlos totalmente.
    \end{enumerate}

        \item ¿Puedes mencionar tres cosas que no salieron bien durante este proyecto?
        \begin{enumerate}
            \item Antonio Cabrera: Solo puedo encontrar dos cosas que no salieron del todo bien: la primera vez que hicimos las épicas no salieron bien del todo; y la segunda cosa que se puede mejorar es la constancia de trabajo ya que algunas semanas trabajamos mucho y otras menos.
            \item Antonio Pérez: La constancia en el trabajo, algunas semanas trabajamos demasiado en los proyectos de la asignatura, lo que llevó a que las siguientes semanas no tuviéramos tantas ganas de realizar el trabajo; la organización, no nos hemos organizado desde un primer momento; y no comenzar con tiempo el prototipo de la aplicación.
            \item Alejandro Jiménez: Desde un principio no hemos sido constantes, lo que ha generado épocas de gran cantidad de trabajo; la realización de las historias de usuarios, al principio no teníamos claro cómo crearlas pero conseguimos solucionarlo; y el prototipo, nos atrasamos con las demás partes lo que llevó a no terminarlo.
        \end{enumerate}

    \item ¿Tienes alguna sugerencia sobre cómo podemos mejorar para el próximo proyecto?
    \begin{enumerate}
        \item Antonio Cabrera: Podríamos agendar reuniones cada dos días para mantener una constancia adecuada.
        \item Antonio Pérez: Aplicar una metodología ágil desde un primer momento (como la que proponemos nosotros), para organizarnos mejor; y tomar pausas al realizar los trabajos, para no estar quemados en las siguientes tareas.
        \item Alejandro Jiménez: Se podrían aplicar metodologías como SCRUM para poder llevar mejor el proyecto sin que haya grandes acumulaciones de trabajo.
    \end{enumerate}
    
    \item ¿Te gustaría volver a trabajar en un proyecto como este?¿Por qué o por qué no?
    \begin{enumerate}
        \item Antonio Cabrera: Me gustaría más trabajar en un trabajo práctico ya que este era muy teórico.
        \item Antonio Pérez: Si, me gusto mucho el ambiente de trabajo, además, me ha gustado realizar un proyecto desde cero. Y tengo ganas de aplicar desde un primer momento estrategias de trabajo que hemos ido descubriendo.
        \item Alejandro Jiménez: La realización de este proyecto nos ha dado una perspectiva nueva de cómo se realizan los trabajos y no como habíamos hecho con anterioridad, pero me gustaría trabajar más en una parte práctica.
 
    \end{enumerate}

    \item Después de esta experiencia, si pudieras volver a hacer este proyecto, ¿qué harías de
    forma diferente?
    \begin{enumerate}
        \item Antonio Cabrera: Intentaría hacer una prueba de concepto del proyecto funcionando de verdad.
        \item Antonio Pérez: Empezaría con el desarrollo del prototipo mucho antes, en vez de esperar a que el resto de apartados estén bien pulidos, de esta forma podríamos tener un feedback del cliente a medida que avanzamos con la aplicación.
        \item Alejandro Jiménez: Desde un principio realizaría una agenda para no acumular trabajo. 
    \end{enumerate}
\end{enumerate}


