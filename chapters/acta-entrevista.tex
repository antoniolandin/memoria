\chapter{Actas de entrevistas}

\section{Acta 16/10/2024}

\textbf{ACTA DE LA ENTREVISTA REALIZADA AL CLIENTE PARA RESOLVER DUDAS SOBRE SU PETICIÓN. K-VATA, PARA EL DESARROLLO DE UN PROYECTO DE GESTIÓN Y VENTA DE RECURSOS.} \\

En U-tad, a dieciséis de octubre de dos mil veinticuatro, siendo las 18:20 horas, se reúnen los siguientes integrantes de la empresa K-VATA para realizar la entrevista al cliente de la bodega.

\begin{itemize}
    \item Antonio Pérez Mérquez
    \item Antonio Cabrera Landín
    \item Alejandro Jiménez García
\end{itemize}

En la entrevista se realizaron las siguientes preguntas con la respuesta correspondiente:

\begin{enumerate}
    \item ¿Si hay algún error en el inicio de sesión se bloquea o se realiza alguna otra acción?

    \begin{enumerate}
        \item Al iniciar sesión cada uno de los empleados introduce su usuario y
contraseña. Además, si un empleado se equivoca al introducir sus
credenciales no habrá restricciones, es decir, los intentos serán ilimitados.
    \end{enumerate}

    \item ¿Cómo acceden los administradores a CoreBodegas?
    \begin{enumerate}
        \item Los administradores de la bodega (gerente de la bodega) accederán como todos los usuarios, sin embargo, serán los únicos que puedan acceder a la parte de personal.
    \end{enumerate}

    \item ¿Existen productos predeterminados o se introducen todos los datos?
    \begin{enumerate}
        \item Todos los productos deben ser introducidos por lo menos una vez.
    \end{enumerate}

    \item ¿Qué datos hay que añadir para ingresar productos?
    \begin{enumerate}
        \item Cuando se agregue un producto se debe introducir un ID, una descripción, el precio (tanto en euros como en dólares), el formato y la cosecha.
    \end{enumerate}

    \item ¿Los productos se pueden eliminar?
    \begin{enumerate}
        \item Los datos se podrán ajustar, es decir, se podrán añadir o reducir la cantidad de producto. Pero no se podrá eliminar el producto (su cantidad mínima es 0).
    \end{enumerate}

    \item ¿Se necesita algún tipo de validación o confirmación para ingresar datos?
    \begin{enumerate}
        \item Tras ingresar los datos se podrán guardar a través de un botón que lo permite y a su vez lo valide.
    \end{enumerate}

    \item ¿Qué datos del cliente/empresa se necesitan aparte del domicilio?
    \begin{enumerate}
        \item El cliente consta de un ID, nombre, apellido, NIF, dirección, teléfono y fecha de cumpleaños.
        \item Las empresas constan de nombre, CIF, dirección, teléfono, dirección de entrega y descuento.
    \end{enumerate}

    \item ¿Se requiere de una confirmación para eliminar o modificar algún elemento?
    \begin{enumerate}
        \item  Todos los datos que se agreguen o eliminen tiene un botón para guardar los cambios.
    \end{enumerate}

    \item ¿Cuáles son los datos introducidos en el pedido?
    \begin{enumerate}
        \item Los datos necesarios para un pedido son los siguientes: ID, ID del cliente, estado (trámite, cerrado, entregado y liquidado) y los detalles.
    \end{enumerate}

    \item ¿Al realizar las modificaciones que tipo de confirmación es necesaria?
    \begin{enumerate}
        \item Para cualquier tipo de modificación se requiere de un botón para aceptar los cambios y validarlos
    \end{enumerate}

    \item ¿Qué tipos de datos se añaden en la factura a parte de la cantidad y tipo de producto?
    \begin{enumerate}
        \item Los datos que deben ser introducidos en la factura serán los mismos que en el pedido más la dirección de entrega y nombre del cliente.
    \end{enumerate}

    \item ¿Se necesitan variables extras?
    \begin{enumerate}
        \item Se requiere de secciones que diferencien entre los productos que ha pedido el cliente.
    \end{enumerate}

    \item ¿Qué datos del pedido se necesitan?
    \begin{enumerate}
        \item Se necesitan los detalles del pedido. Que consta de cantidad de botellas, el tipo de producto, precio por producto, importe total del pedido con IVA y sin IVA
    \end{enumerate}

    \item ¿Se imprime la factura digital y se facilita al cliente?
    \begin{enumerate}
        \item La factura siempre se enviará al cliente. Además, dentro de la sección de productos existirá un botón para imprimir la factura.
    \end{enumerate}

    \item ¿Se registran las facturas a los clientes del pedido?
    \begin{enumerate}
        \item La factura se deberá guardar asociado al ID del cliente.
    \end{enumerate}
\end{enumerate}

\section{Acta 06/11/2024}

\textbf{ACTA DE LA ENTREVISTA REALIZADA AL CLIENTE PARA RESOLVER DUDAS SOBRE SU PETICIÓN. K-VATA, PARA EL DESARROLLO DE UN PROYECTO DE GESTIÓN Y VENTA DE RECURSOS.}\\

En U-Tad, a seis de noviembre de dos mil veinticuatro, siendo las 17:30 horas, se reúnen los siguientes integrantes de la empresa K-VATA para realizar la entrevista al cliente de la bodega.
\begin{itemize}
    \item Alejandro Jiménez García
\end{itemize}

En la entrevista se realizaron las siguientes preguntas con la respuesta correspondiente:

\begin{enumerate}
    \item ¿Con respecto a la última vez se ha pensado algún cambio sobre el proyecto?
    \begin{enumerate}
        \item Los clientes deben estar incluidos en la información del pedido.
    \end{enumerate}

    \item ¿Las reservas de productos deben realizarse de alguna manera en específica?
    \begin{enumerate}
        \item Al realizar la compra se guarda el pedido y se reduce la cantidad del producto pedido.
    \end{enumerate}

    \item ¿Existe la prioridad en los pedidos?
    \begin{enumerate}
        \item Los pedidos se tramitarán en el orden en que se hayan recibido.
    \end{enumerate}

    \item Para usted cuál es el orden de prioridad de las tareas.
    \begin{enumerate}
        \item 1. Gestión de Empleados/Materia 2. Cliente 3. Pedidos 4.Productos
    \end{enumerate}

    \item ¿Hay alguna calificación más que quiera añadir a la uva? Por ejemplo, tipo de uva.
    \begin{enumerate}
        \item El tipo de uva debería estar incluido, a parte de eso ningún dato más. 
    \end{enumerate}
\end{enumerate}
