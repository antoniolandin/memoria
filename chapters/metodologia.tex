\chapter{Metodología de la empresa}

\section{Investigación previa}

\subsection{Metodologías Ágiles}

\subsubsection{Scrum}

Es una forma de organizar el trabajo en ciclos cortos denominados sprints, estos ciclos suelen durar de entre 2 a 4 semanas, con la idea de aumentar la calidad del producto en cada entrega o añadir funcionalidades nuevas.\\

Dentro de esta metodología existen varios roles importantes Scrum Master (facilita los procesos y elimina obstáculos), Product Owner (gestiona el backlog y prioriza tareas) y Equipo de Desarrollo (Realiza el trabajo técnico).\\

Además, tiene ciertas características que lo distingue, como los distintos tipos de reuniones Daily Scrum (), Sprint Planning (para planificar el próximo ciclo de Sprint) y Sprint Review (tras terminar el sprint).\\

Funciona muy bien en proyectos con requisitos cambiantes.

\subsubsection{Kanban}

Un método visual de gestión de flujo de trabajo que utiliza un tablero Kanban para rastrear las tareas desde “Por Hacer” hasta “Terminado” (como un trello). Este método tiene el objetivo de limitar el trabajo en proceso, evitando la sobrecarga de este, y mejorar continuamente los procesos. Por otro lado, no hacen falta iteraciones fijas. Funciona muy bien en equipos con tareas recurrentes o impredecibles

\subsubsection{Safe}

Una forma de implementar las metodologías ágiles en empresas grandes, consiguiendo una coordinación entre varios equipos. Este método implementa el Scrum o/y el Kanban, ya que une varios equipos ágiles que usan estas metodologías, aplicando dentro de estos las mismas características que se han destacado antes.\\

Con este método se agrupan varios equipos en un Agile Release Train, que se define como un “equipo de equipos” con el objetivo de entregar incrementos de gran valor. Se trabaja en Program Increments, que son ciclos de entre 8 a 12 semanas con múltiples sprints dentro de este.\\

Funciona muy bien con empresas grandes de miles de personas.

\subsubsection{DevOps}

Combina el desarrollo con las operaciones para automatizar y optimizar el ciclo de vida del software.\\

Cuenta con unos principios a seguir Integración continua (Automatiza la integración de código), Entrega continua (Automatiza el despliegue) y Colaboración (Incita la comunicación entre equipos). Esta práctica se lleva a cabo con el uso de herramientas como Jenkins, Dockers y Kubernetes, necesitando una monitorización constante y feedback rápido.\\

Funciona muy bien en equipos con tiempo limitado.

\subsubsection{Lean Development}

Diseñada para optimizar el desarrollo de software. Tiene un enfoque en eliminar cualquier desperdicio, maximizar el valor del producto entregado al cliente y mejorar el proceso.\\

Con esta metodología se busca reducir el número de tareas innecesarias, el tiempo perdido y la duplicación de los esfuerzos; Previene errores y el diseño es óptimo desde un primer momento; busca el aprendizaje a lo largo del desarrollo; valora la entrega del producto rápido (ciclos cortos); Fomenta la colaboración entre el equipo y la autonomía de este; Mejorar el flujo general, no se centra en partes individuales del producto.\\

Funciona muy bien en equipos que buscan adaptabilidad y eficiencia.

\subsubsection{Extreme Programming (XP)}

Prioriza las necesidades del cliente, centrándose en la calidad de software.\\

Promueve la colaboración cliente-equipo, llevando a toma de decisiones rápidas. Incluyendo prácticas técnicas como integración continua y programación en pareja. Además de buscar un ambiente de trabajo respetuoso y sin miedo de tomar decisiones difíciles.\\

Prácticas de esta metodología:
\begin{itemize}
    \item  Desarrollo guiado por pruebas
    \item Programación en parejas
    \item Integración continua
    \item Pequeñas entregas
    \item Refactorización constante (Limpiar el código)
    \item Retroalimentación frecuente del cliente
    \item Simplicidad del diseño
    \item Propiedad colectiva del código
    \item Ritmo sostenible
    \item Pruebas funcionales
    \item Coding standards
\end{itemize}\\

Funciona muy bien con equipos pequeños.

\subsection{Metodologías de empresas}

En este apartado, estudiaremos algunas de las metodologías ágiles que utilizan empresas líderes en el ámbito tecnológico.

\subsubsection{Amazon}

Trabajan con un sistema de Working Backwards, comienzan pensando en la experiencia del cliente, por lo que hacen un comunicado de prensa y un FAQ antes del desarrollo. Una vez hecho el comunicado se basan en todos los datos que tienen para tomar una decisión, por lo que usan herramientas como A/B Testing.\\

Sus equipos siguen el concepto de 2 pizza teams, es decir, equipos pequeños, autónomos y multidisciplinarios, son equipos que pueden ser alimentados por dos pizzas. Al ser tan pequeños se tiene una toma de decisiones mucho más rápida.\\

A la hora de las memorias de funcionalidades, rechazan todo lo que son presentaciones y presentan documentos de seis páginas, para fomentar la claridad y profundidad.

\subsubsection{Spotify}

Trabajan con una metodología propia a la que denominan, Spotify Model. Este modelo organiza equipos de forma descentralizada, para ello se asignan Squads (equipos pequeños) que forman lo que llaman Tribes (grupos de squads que tienen un objetivo relacionado), y los Guilds (comunidades informales). También existen los chapters (grupos de personas con habilidades similares en diferentes squads).\\

Además, se usan los principios de Lean Startup y DevOps.

\section{Metodología de nuestra empresa}

Viendo todas las metodologías más usadas actualmente, nos hemos dado cuenta que no existe ninguna que junte metodologías administrativas con las de implementación práctica, por lo que vamos a crear una que cumpla con esto y que además tenga en cuenta el bienestar de nuestros empleados.\\

Lo primero es definir la estructura de los equipos, ya que son la parte más fundamental de un proyecto, sin los equipos sería muy complicado organizar a grandes grupos de personas, por ello hemos decidido usar dos tipos:
\begin{itemize}
    \item Grupos: Son equipos formados por pocas personas (4-10) que están encargados de una pequeña parte del proyecto.
    \item Clanes: Un conjunto de grupos con un mismo objetivo conjunto, es decir, trabajan en una misma parte del proyecto.
\end{itemize}

El método a seguir para la planificación del proyecto está dividida en varias partes, cada una de estas partes es importante para un seguimiento acertado del proyecto y poder estimar el tiempo necesario para realizar próximas partes:
\begin{itemize}
    \item Sprints: Son periodos de entre 1 a 3 meses, durante este tiempo se intentará completar en su totalidad un objetivo definido en la reunión de comienzo de sprint.
    \item Sub-Sprints: Son periodos de entre 1 a 2 semanas y conforman un sprint, al comienzo de este se especificarán las user stories que queremos completar, para ello nos reuniremos con el cliente quien priorizará cada una de las US seleccionadas y el equipo de desarrollo estimará el esfuerzo que llevará dicha tarea.
    \item Daily chat: Es una reunión diaria de unos 20-30 mins donde se discutirán problemas que hayan surgido a lo largo de ese día y lo que se va a avanzar el próximo.
    \item Emergencia: En caso de que se haya infravalorado una US y requiera más esfuerzo del establecido en la reunión del sub-sprint, se llamará a una reunión de emergencia donde junto al cliente discutiremos cómo solventar el obstáculo.
    \item Evaluation: Al finalizar un sub-sprint se hará una reunión con todo el equipo de desarrollo para exponer todos los avances realizados y prepararse para el siguiente sub-sprint.
    \item End: Una vez finalizado el sprint se establecerá una reunión con el cliente para enseñarle los avances conseguidos, verificar su valor y discutir qué camino tomar en el próximo sprint.
\end{itemize}

Una vez explicada la forma en la que se va a planificar cada uno de los grupos, expondremos las implementaciones prácticas para mejorar la calidad y el valor del producto entregado, además de mantener al cliente cerca del desarrollo, evitando futuras discusiones:
\begin{itemize}
    \item En las reuniones de los sub-sprints contaremos con la presencia del cliente, el se encargará de asignar el valor que tiene cada una de US, de esta forma estaremos completamente seguros de la importancia de cada una de ellas, además de contar con la estimación de esfuerzo de los desarrolladores, esto facilitará el no asignar US inviables o demasiado complejas.

    \item Antes de comenzar la implementación de código, los desarrolladores definirán una serie de tests automatizados que el software deberá completar correctamente al finalizar su desarrollo, esto ayudará a conseguir una mejor funcionalidad de la aplicación y es una forma sencilla de poder comprobar el código.

    \item A la hora del diseño, se mantendrá un diseño lo más simple posible a lo largo del desarrollo para centrarnos principalmente en la calidad del código y su funcionalidad, una vez terminado el software se modificará el diseño de este para tener más estética.

    \item Al implementar código todos los desarrolladores pueden editar todas las partes del software, por lo que todos pueden editar cualquier zona, esto fomenta una responsabilidad compartida hacia el código, haciendo más eficaz la producción de este.
\end{itemize}

Otra parte importante será la transparencia dentro del proyecto, todas las actividades estarán visibles para todo el mundo en un único tablero, dentro de este se dispondrá de toda la información necesaria para llevar a cabo la tarea y quien la comprenderá, para poder ver todo de una forma más visual existen varios iconos que se les asignará a cada US para transmitir rápidamente información a los miembros del grupo.\\

Por último, queremos que cada miembro de cada grupo mantenga su bienestar, para esto se tendrá en cuenta cómo se siente cada uno de los integrantes emocionalmente. Se fomentan las pausas activas, es decir, descansos del trabajo a lo largo del día en unas zonas designadas para ello donde se contará con varios juegos o zonas para socializar con otros grupos, de esta forma se alivia el estrés y el malestar. Por otro lado, los miembros de la empresa contarán con la posibilidad de proponer actividades grupales o participar en ellas, donde sin importar tu grupo o clan podrás desarrollar todavía más tu carrera, aprendiendo nuevas tecnologías, estrategias, etc. Para terminar, se dispondrá de un psicólogo para mantener la salud mental de los empleados, estos podrán establecer citas cuando las necesiten.
